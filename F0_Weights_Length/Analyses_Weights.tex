\documentclass[]{article}
\usepackage{lmodern}
\usepackage{amssymb,amsmath}
\usepackage{ifxetex,ifluatex}
\usepackage{fixltx2e} % provides \textsubscript
\ifnum 0\ifxetex 1\fi\ifluatex 1\fi=0 % if pdftex
  \usepackage[T1]{fontenc}
  \usepackage[utf8]{inputenc}
\else % if luatex or xelatex
  \ifxetex
    \usepackage{mathspec}
  \else
    \usepackage{fontspec}
  \fi
  \defaultfontfeatures{Ligatures=TeX,Scale=MatchLowercase}
\fi
% use upquote if available, for straight quotes in verbatim environments
\IfFileExists{upquote.sty}{\usepackage{upquote}}{}
% use microtype if available
\IfFileExists{microtype.sty}{%
\usepackage{microtype}
\UseMicrotypeSet[protrusion]{basicmath} % disable protrusion for tt fonts
}{}
\usepackage[margin=1in]{geometry}
\usepackage{hyperref}
\hypersetup{unicode=true,
            pdftitle={Big\_Experiment\_Analyses},
            pdfauthor={Hamza},
            pdfborder={0 0 0},
            breaklinks=true}
\urlstyle{same}  % don't use monospace font for urls
\usepackage{longtable,booktabs}
\usepackage{graphicx,grffile}
\makeatletter
\def\maxwidth{\ifdim\Gin@nat@width>\linewidth\linewidth\else\Gin@nat@width\fi}
\def\maxheight{\ifdim\Gin@nat@height>\textheight\textheight\else\Gin@nat@height\fi}
\makeatother
% Scale images if necessary, so that they will not overflow the page
% margins by default, and it is still possible to overwrite the defaults
% using explicit options in \includegraphics[width, height, ...]{}
\setkeys{Gin}{width=\maxwidth,height=\maxheight,keepaspectratio}
\IfFileExists{parskip.sty}{%
\usepackage{parskip}
}{% else
\setlength{\parindent}{0pt}
\setlength{\parskip}{6pt plus 2pt minus 1pt}
}
\setlength{\emergencystretch}{3em}  % prevent overfull lines
\providecommand{\tightlist}{%
  \setlength{\itemsep}{0pt}\setlength{\parskip}{0pt}}
\setcounter{secnumdepth}{0}
% Redefines (sub)paragraphs to behave more like sections
\ifx\paragraph\undefined\else
\let\oldparagraph\paragraph
\renewcommand{\paragraph}[1]{\oldparagraph{#1}\mbox{}}
\fi
\ifx\subparagraph\undefined\else
\let\oldsubparagraph\subparagraph
\renewcommand{\subparagraph}[1]{\oldsubparagraph{#1}\mbox{}}
\fi

%%% Use protect on footnotes to avoid problems with footnotes in titles
\let\rmarkdownfootnote\footnote%
\def\footnote{\protect\rmarkdownfootnote}

%%% Change title format to be more compact
\usepackage{titling}

% Create subtitle command for use in maketitle
\newcommand{\subtitle}[1]{
  \posttitle{
    \begin{center}\large#1\end{center}
    }
}

\setlength{\droptitle}{-2em}

  \title{Big\_Experiment\_Analyses}
    \pretitle{\vspace{\droptitle}\centering\huge}
  \posttitle{\par}
    \author{Hamza}
    \preauthor{\centering\large\emph}
  \postauthor{\par}
      \predate{\centering\large\emph}
  \postdate{\par}
    \date{1 June 2019}


\begin{document}
\maketitle

\section{Boxplot}\label{boxplot}

Comparing between sexes and treatment groups over time. The treatment
groups continue to gain more weight, and as expected, females are always
heavier (in both groups)
\includegraphics{Analyses_Weights_files/figure-latex/unnamed-chunk-3-1.pdf}

\section{Individual growth}\label{individual-growth}

I thought it would be intersting to see the indvidual growth
trajectories. I suspect the zebrafish hierachy is coming into play. Some
fish in the control group are still getting obese quick because they're
dominating; while some fish in the treatment group aren't gaining weight
as fast because they're subdued.
\includegraphics{Analyses_Weights_files/figure-latex/unnamed-chunk-4-1.pdf}

\section{Summary}\label{summary}

\begin{longtable}[]{@{}rllrrrrr@{}}
\caption{Summary}\tabularnewline
\toprule
Week & Tank & Treatment\_Tank & N & Weight.g. & sd & se &
ci\tabularnewline
\midrule
\endfirsthead
\toprule
Week & Tank & Treatment\_Tank & N & Weight.g. & sd & se &
ci\tabularnewline
\midrule
\endhead
0 & HA\_C1 & Control & 24 & 0.2214583 & 0.0457754 & 0.0093439 &
0.0193293\tabularnewline
0 & HA\_C2 & Control & 24 & 0.2255833 & 0.0493646 & 0.0100765 &
0.0208448\tabularnewline
0 & HA\_C3 & Control & 24 & 0.2363333 & 0.0486046 & 0.0099214 &
0.0205239\tabularnewline
0 & HA\_C4 & Control & 24 & 0.2240833 & 0.0487406 & 0.0099491 &
0.0205813\tabularnewline
0 & HA\_C5 & Control & 24 & 0.2115417 & 0.0510703 & 0.0104247 &
0.0215651\tabularnewline
0 & HA\_T1 & Treatment & 24 & 0.2489583 & 0.0484736 & 0.0098946 &
0.0204686\tabularnewline
\bottomrule
\end{longtable}

\section{Line plot}\label{line-plot}

\includegraphics{Analyses_Weights_files/figure-latex/unnamed-chunk-6-1.pdf}

\section{Mixed Model}\label{mixed-model}

~

Weight g

Predictors

Estimates

CI

p

(Intercept)

0.25

0.24~--~0.26

\textless{}0.001

Treatment\_TankTreatment

0.04

0.03~--~0.05

\textless{}0.001

Week

0.04

0.04~--~0.04

\textless{}0.001

Sexmale

-0.08

-0.09~--~-0.07

\textless{}0.001

Random Effects

σ2

0.00

τ00 Fish\_ID

0.00

ICC Fish\_ID

0.58

Observations

720

Marginal R2 / Conditional R2

0.647 / 0.852

The results indicate the treatment is working, and as expected, females
are significantly heavier. It's interesting that the control group is
still gaining weight. I assume this will even out soon enough while the
treatment group continues to gain weight. Every week we are seeing
significant changes in weight.


\end{document}
